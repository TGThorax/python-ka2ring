% -*- mode: Latex; ispell-dictionary: "dutch"; -*-

\section*{Collecties en lussen}
  Stukken data zijn soms gemakkelijker om mee te werken als ze gegroepeerd zijn.
  Er bestaan heel erg veel soorten groeperingen of collecties. Een vaak
  gebruikte soort collectie is een lijst. Python maakt het ons dan ook
  gemakkelijk om met lijsten te werken.

\section{Lijsten}
  \phantomsection
  \addcontentsline{toc}{subsection}{Aanmaken}
  In Python maak je lijsten met vierkante haakjes \py{[\,]}. Voor de splitsing
  tussen elementen in een lijst gebruik je komma's:
  \begin{python}
    toespijs = ["kaas", "ham"]
    groenten = ["sla", "tomaat", "augurk"]
  \end{python}
  Je kan lijsten samenvoegen net zoals je dat met strings kan:
  \begin{python}
    ingredienten = toespijs + groenten
  \end{python}

  \phantomsection
  \addcontentsline{toc}{subsection}{Indexeren}
  Lijsten zijn \emph{ge\"indexeerd} met nummer, wat wil zeggen dat elk element
  in een lijst een nummer krijgt. Je kan dan met die nummers de elementen
  opvragen in de lijst. Let wel: nummering of \emph{indexering} start vanaf $0$.
  Het eerste element in een lijst is voor Python het element met nummer of
  \emph{index} $0$, het tweede is dat met index $1$, enzovoort. Indexeren doe je
  eveneens met vierkante haakjes \py{[\,]}:
  \begin{python}
    toespijs[0]   #=> "kaas"
    groenten[2]   #=> "augurk"
  \end{python}
  Als je Python vraagt naar een element met een index die buiten het aantal
  elementen in de lijst valt, zal Python reclameren. Hij zegt dat de index niet
  binnen het bereik van de lijst ligt:
  \begin{python}
    toespijs[2]   #= IndexError: list index out of range
  \end{python}
  Om te weten hoeveel elementen er in de lijst zitten, kan je altijd de functie
  \py{len} gebruiken:
  \begin{python}
    len(toespijs) #=> 2
  \end{python}
  De indexering is ook cyclisch: je kan negatieve getallen gebruiken om
  achteraan te beginnen. Zo is $-1$ het laatste element, $-2$ het op \'e\'en na
  laatste, \ldots:
  \begin{python}
    groenten[-1] == groenten[len(groenten) - 1] #=> True
  \end{python}
  Je kan ook met indexering de elementen toewijzen en verwijderen.
  Verwijderen doe je met \py{del}:
  \begin{python}
    groenten[2] = "komkommer"
    groenten  #=> ["sla", "tomaat", "komkommer"]

    del toespijs[0]
    toespijs  #=> ["ham"]
  \end{python}
  Noot: \py{del} is niet specifiek aan lijsten. Je kan net zo goed een variabele
  verwijderen: \py{del groenten}.

  Let wel: als je iets verandert in de ene lijst, is dat niet per se zo in de
  andere. Zo is onze \py{ingredienten}-lijst nog steeds gelijk aan \py{["kaas",
    "ham", "sla", "tomaat", \char`"augurk"]}, ookal hebben we
  \py{\char`"augurk"} met \py{"komkommer"} vervangen in \py{groenten} en
  \py{"kaas"} verwijderd in \py{toespijs}. Hoe dit juist werkt zie je in het
  extra gedeelte \ref{sec:pointers}.

  \phantomsection
  \addcontentsline{toc}{subsection}{Versnijden of slicen}
  Je kan ook lijsten versnijden, of \emph{slicen}. Daarmee krijg je een kopie
  van je lijst voor een bepaald gedeelte. Slicen doe je met het karakter \py{:}
  binnenin de rechte haakjes van de indexering.
  \begin{python}
    lijst = list(range(1, 11))
    lijst            #=> [1, 2, 3, 4, 5, 6, 7, 8, 9, 10]
    lijst[2:4]       #=> [3, 4]
    lijst[5:-2]      #=> [6, 7, 8]
    lijst[9:-1]      #=> []  # De lege lijst, dus.
    lijst[:3] == lijst[0:3]  #=> True
    lijst[7:]        #=> [8, 9, 10]

    lijst[-2:] = [1, 2]
    del lijst[:7]
    lijst            #=> [8, 0, 1]
  \end{python}
  Het vorige voorbeeld maakt gebruik van de functie \py{range} die een reeks
  maakt met opeenvolgende getallen. Met \'e\'en argument, \py{range(n)}, krijg
  je een reeks van $0$ tot (niet met) $n$; met twee argumenten, \py{range(n,
    m)}, krijg je een reeks van $n$ tot $m$. Let wel, deze reeks is nog geen
  lijst. Het is een andere soort verzameling of collectie: de \emph{range} of
  \emph{reeks}.
  Om zo'n andere soort collectie naar een lijst om te vormen gebruik je de
  functie \py{list} zoals in het voorbeeld. Een reeks is heel erg gelijkaardig
  aan een lijst. Het enige grote verschil is dat je in een reeks geen dingen mag
  veranderen, terwijl je dat in een lijst wel mag. Indexering en slicing werkt
  volledig hetzelfde. Natuurlijk zit er in een reeks ook de structuur van
  opeenvolgende getallen (met mogelijk een stap tussen getallen). Zo'n structuur
  is niet nodig in een lijst.

  \phantomsection
  \addcontentsline{toc}{subsection}{Algemene operaties}
  Met \py{in} kan je zien of een element aanwezig is in een lijst:
  \begin{python}
    if "augurk" in groenten:
      print("Sorry, er zijn geen augurken meer.")
  \end{python}

  Een lijst heeft ook een bepaalde vorm van waarheid ingebakken. Zo is een lege
  lijst vals-achtig en een lijst met elementen waar-achtig:
  \begin{python}
    if not (toespijs or groenten):
      print("Je wilt een broodje zonder iets op?!")
  \end{python}
  Om te zien wat je nog allemaal kan doen met lijsten kan je \py{dir([])}
  uitvoeren. Zo vraag je aan Python ``Wat kan ik allemaal doen met een lege
  lijst?'' Zo zal je zien dat een functie \py{append} bestaat die \'e\'en item
  toevoegt aan een lijst. De syntax om zo'n functie te gebruiken is:
  \begin{itemize}
  \item de variabele of object waarop je het wil gebruiken
  \item een punt (\py{.})
  \item de naam van de functie met argumenten
  \end{itemize}
  Bijvoorbeeld:
  \begin{python}
    groenten.append("wortel")
    groenten #=> ["sla", "tomaat", "komkommer", "wortel"]
  \end{python}
  Natuurlijk konden we net zo goed \py{groenten += ["wortel"]} doen (maar
  \py{append} is een tikkeltje sneller en werkt ook anders. Zie sectie
  \ref{sec:pointers}).

\section{Strings}
  Eigenlijk ken je nog een andere soort collectie: de string. Een string is een
  geordende verzameling van karakters. Het lijkt heel erg op een lijst. Ze zijn
  onveranderlijk, net zoals de reeks. Verder kan je ook veel meer
  tekst-specifieke functies gebruiken. Gebruik \py{dir(\char`"\char`")} voor
  meer info.


\section{Lussen}
% TODO
  \subsection{for .. in:}
  Soms wil je dat je programma iets herhaaldelijk doet \emph{voor elk} element
  \emph{in} een lijst. Daarvoor gebruiken we de \emph{for}-lus. Het volgende
  programma illustreert dit. Het print alle getallen van $0$ tot en met $100$:
  \begin{python}
    for getal in range(101):
      print(getal)
  \end{python}
  Zoals je ziet zal Python \'e\'en voor \'e\'en de getallen in de reeks aan de
  variabele \py{getal} toewijzen. Daarna kan je dat gewoon in de blok eronder
  gebruiken als een variabele. Ook na de lus is \py{getal} nog een gekende
  variabele, maar natuurlijk zal Python die niet meer zomaar veranderen. De
  waarde in de variabele is de laatst gebruikte waarde in de lus. In dit
  geval is \py{getal} na de lus gelijk aan $100$.

  Je hoeft natuurlijk helemaal niet met getallen te werken in de lijsten of
  reeksen. We kunnen net zo goed met bijvoorbeeld strings werken:
  \begin{python}
    # We zijn zeker zijn dat len(ingredienten) >= 2
    bestelling = "Voor mij een broodje met "
    for ingr in ingredienten[:-2]:
      bestelling += ingr + ", "
    bestelling += "en " + ingredienten[-1] + " graag."
  \end{python}

  \subsection{while}. Soms willen we dat ons programma iets blijft herhalen
  \emph{zolang} iets geldt. Daarvoor gebruiken we \py{while}. Het volgende
  programma blijft oneindig lang het getal $1$ printen:
  \begin{python}
    while True:
      print(1)
  \end{python}
  Hierboven was de test natuurlijk altijd waar (\py{True}). Je kan natuurlijk
  ook een voorwaarde geven die niet per se altijd waar is. Bijvoorbeeld:
  \begin{python}
    n_broden = 10
    while (n_broden > 0):         # Zolang er broden zijn
      antwoord = int(input("Hoeveel broden wil je? "));
      if (antwoord < 0):          # We krijgen broden
        print("Ik ben hier wel de bakker, niet jij!")
      elif (antwoord > n_broden): # Er zijn te weinig broden
        print("Sorry, er zijn maar " + \
          str(n_broden) + " meer.")
      else:
        print("Alsjeblieft!")
        n_broden -= antwoord
  \end{python}
  Hierboven gebruikten we ook het karakter \py{\textbackslash} om aan Python
  aan te geven dat de volgende lijn eigenlijk het vervolg is op de huidige
  lijn.

  \subsection{break en continue}. Je kan deze lussen ook vroegtijdig afbreken
  of terug van bovenaan beginnen. Dat doe je door \py{break} en \py{continue}
  respectievelijk in je lus te zetten. Dat is natuurlijk enkel handig als je dat
  voorwaardelijk doet. Tenzij je echt te veel condities hebt binnen de lus om het
  nog gemakkelijk te kunnen begrijpen, kan je ze best zo weinig mogelijk te
  gebruiken.

\section{Verwijzingen (extra)}\label{sec:pointers}
  Neem de volgende code:
  \begin{python}
    a = [1, 2]
    b = a
  \end{python}
  Wanneer Python dit uitvoert wordt eerst het stukje data \py{[1, 2]}
  aangemaakt. Daarna wordt de variabele \py{a} een \emph{verwijzing} naar dat
  stuk data. Als Python dan \py{b = a} tegenkomt, dan wordt ervoor gezorgd dat
  de variabele \py{b} naar dezelfde data verwijst als waar \py{a} naar verwijst.
  Zo is het duidelijk dat als we iets in \py{a} aanpassen, we het dan ook in
  \py{b} aanpassen en vice-versa, aangezien beide variabelen naar dezelfde data
  verwijzen.

  Vergelijk met het volgende:
  \begin{python}
    a = [1, 2]
    b = a + [3, 4]
  \end{python}
  Opnieuw wordt eerst \py{[1, 2]} aangemaakt en wordt a een verwijzing daarnaar.
  Bij de tweede lijn gebeurt er echter iets anders als voorheen. Eerst wordt de
  optelling/samenvoeging door Python geregeld. Daardoor krijg je een
  \emph{nieuw} stukje data \py{[1, 2, 3, 4]}. Dan pas wordt \py{b} een
  verwijzing naar die data. Aan de verwijzing in \py{a} wordt niet geraakt. De
  variabelen \py{a} en \py{b} verwijzen nu naar twee verschillende stukken data.
  Het is dan logisch dat als je iets in het ene stukje data aanpast, je dat niet
  in het andere doet.

  Wat als je wilt dat alle verwijzingen aangepast worden als je elementen
  toevoegt? Zoals je ziet kunnen we \py{+} niet gebruiken, want dat zorgt voor
  nieuwe data. Wat we wel kunnen doen is de lijst-functies \py{append} en
  \py{extend} gebruiken:
  \begin{python}
    a = [1, 2]
    b = a
    a.append(3)
    b.extend([4, 5])
  \end{python}
  De tweede lijn zorgt ervoor dat \py{a} en \py{b} naar hetzelfde verwijzen,
  zoals voorheen. De functies op de laatste twee lijnen \emph{werken in op de
    data zelf}, dus veranderen niets aan de verwijzingen \py{a} en \py{b} maar
  wel aan de data waarnaar ze verwijzen. Op het einde verwijzen beide variabelen
  dus naar hetzelfde stuk data \py{[1, 2, 3, 4, 5]}.

  Iedereen die een verwijzing heeft naar je data kan die ook aanpassen. Dat kan
  zowel een zorg als een zegen zijn. Om de zorgen een beetje te verlichten
  bestaan er de onveranderlijke data-types. Hiervan kennen we bijvoorbeeld de
  reeks en de string, die ook een collectie zijn. Naast die collecties
  ken je nog onveranderlijke types: de \py{int}, de \py{float} en de \py{bool}.
  In Python kan je dus niet een verwijzing maken naar bijvoorbeeld een getal.
  Als je dat echt per se wel zou willen, kan je ze bijvoorbeeld in een lijst
  zetten.

  Tot nu toe hebben enkel gezien dat variabelen een verwijzing kunnen zijn. Dat
  is niet in het algemeen zo. Zie het volgende voorbeeld:
  \begin{python}
    a = [1, 2]
    b = [a, [3, 4]] # Dus b == [[1, 2], [3, 4]]
  \end{python}
  Nu is het eerste element in de lijst waarnaar \py{b} verwijst zelf een
  verwijzing, namelijk naar \py{[1, 2]}. Dat laatste is ook waar \py{a} naar
  verwijst. Iets aanpassen in \py{a} is dus hetzelfde als iets aanpassen in
  \py{b[0]} en vice-versa.

  Om te weten of twee variabelen naar hetzelfde verwijzen kunnen we \py{is}
  gebruiken:
  \begin{python}
    a = [1, 2, 3]
    b = a
    c = [1, 2, 3]
    a == c    #=> True  # Ze zijn gelijk!
    a is c    #=> False # Maar verwijzen niet naar hetzelfde!
    a is b    #=> True  # a en b doen dat wel.
  \end{python}
  Noot: voor getallen, strings, booleans, \ldots werkt \py{is} vaak toevallig
  net hetzelfde als \py{==}. Dit is echter niet in het algemeen zo voor
  onveranderlijke data! Het geldt bijvoorbeeld niet voor reeksen, of
  voor de integer $1$ en de float $1.0$. Gebruik dus enkel \py{==} om
  onveranderlijke data te vergelijken. Als je toch \py{is} gebruikt wees dan
  absoluut zeker dat je dat niet per ongeluk in de plaats van \py{==} doet.
