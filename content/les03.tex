% -*- mode: Latex; ispell-dictionary: "dutch"; -*-

\section*{Collecties en lussen}
  Stukken data zijn soms gemakkelijker om mee te werken als ze gegroepeerd zijn.
  Er bestaan heel erg veel soorten groeperingen of collecties. Een vaak
  gebruikte soort collectie is een lijst. Python maakt het ons dan ook
  gemakkelijk om met lijsten te werken.

\section{Lijsten}
  Lijsten in Python maak je met vierkante haakjes \py{[\,]}. Voor de splitsing
  tussen elementen in een lijst gebruik je komma's:
  \begin{python}
    toespijs = ["kaas", "ham"]
    groenten = ["sla", "tomaat", "augurk"]
  \end{python}
  Je kan lijsten samenvoegen net zoals je dat met strings kan:
  \begin{python}
    ingredienten = toespijs + groenten
  \end{python}
  Lijsten zijn \emph{ge\"indexeerd} met nummer, wat wil zeggen dat elk element
  in een lijst een nummer krijgt. Je kan dan met die nummers de elementen
  opvragen in de lijst. Let wel: nummering of \emph{indexering} start vanaf $0$.
  Het eerste element in een lijst is voor Python het element met nummer of
  \emph{index} $0$, het tweede is die met index $1$, enzovoort. Indexeren doe je
  met vierkante haakjes \py{[\,]}:
  \begin{python}
    toespijs[0]   #=> "kaas"
    groenten[2]   #=> "augurk"
  \end{python}
  Als je aan Python vraagt om een element met een index die buiten het aantal
  elementen in de lijst valt, zal Python reclameren. Hij zegt dat de index niet
  binnen het bereik van de lijst ligt:
  \begin{python}
    toespijs[2]   #= IndexError: list index out of range
  \end{python}
  Om te weten hoeveel elementen er in de lijst zitten, kan je altijd de functie
  \py{len} gebruiken:
  \begin{python}
    len(toespijs) #=> 2
  \end{python}
  De indexering is ook cyclisch: je kan negatieve getallen gebruiken om
  achteraan te beginnen. Zo is $-1$ het laatste element, $-2$ het op \'e\'en na
  laatste, \ldots:
  \begin{python}
    groenten[-1] == groenten[len(groenten) - 1] #=> True
  \end{python}
  Je kan ook met indexering de elementen veranderen en verwijderen.
  Verwijderen doe je met \py{del}:
  \begin{python}
    groenten[2] = "komkommer"
    groenten  #=> ["sla", "tomaat", "komkommer"]

    del toespijs[0]
    toespijs  #=> ["ham"]
  \end{python}
  Noot: \py{del} is niet specifiek aan lijsten. Je kan net zo goed een variabele
  verwijderen: \py{del groenten}.

  Let wel: als je iets verandert in de ene lijst, is dat niet per se zo in de
  andere. Zo is onze \py{ingredienten}-lijst nog steeds gelijk aan \py{["kaas",
    "ham", "sla", "tomaat", \char`"augurk"]}, ookal hebben we
  \py{\char`"augurk"} met \py{"komkommer"} vervangen in \py{groenten} en
  \py{"kaas"} verwijderd in \py{toespijs}. Hoe dit juist werkt zie je in het
  extra gedeelte \ref{sec:pointers}.

  Je kan ook lijsten versnijden, of \emph{slice}n. Daarmee krijg je een kopie
  van je lijst voor een bepaald gedeelte. Slicen doe je met het karakter \py{:}
  binnenin de rechte haakjes van de indexering.
  \begin{python}
    lijst = list(range(1, 11))
    lijst            #=> [1, 2, 3, 4, 5, 6, 7, 8, 9, 10]
    lijst[2:4]       #=> [3, 4]
    lijst[5:-2]      #=> [6, 7, 8]
    lijst[9:-1]      #=> []  # De lege lijst, dus.
    lijst[:3] == lijst[0:3]  #=> True
    lijst[6:] == lijst[6:-1] #=> True

    lijst[-2:] = [1, 2]
    del lijst[:7]
    lijst            #=> [8, 0, 1]
  \end{python}
  Het vorige voorbeeld maakt gebruik van de functie \py{range} die een reeks
  maakt met opeenvolgende getallen. Met \'e\'en argument, \py{range(n)} krijg je
  een reeks van $0$ tot (niet met) $n$; met twee argumenten, \py{range(n, m)},
  krijg je een reeks van $n$ tot $m$. Let wel, deze reeks is nog geen lijst. Het
  is een andere soort verzameling of collectie: de \emph{range} of \emph{reeks}.
  Om zo'n andere soort collectie naar een lijst om te vormen gebruik je de
  functie \py{list} zoals in het voorbeeld. Een reeks is heel erg gelijkaardig
  aan een lijst. Het enige grote verschil is dat je in een reeks geen dingen mag
  veranderen, terwijl je dat in een lijst wel mag. Indexering en slicing werkt
  volledig hetzelfde. Natuurlijk zit er in een reeks ook de structuur van
  opeenvolgende getallen (met mogelijk een stap tussen getallen). Zo'n structuur
  is niet nodig in een lijst.

  Met \py{in} kan je zien of een element aanwezig is in een lijst:
  \begin{python}
    if "augurk" in groenten:
      print("Sorry, er zijn geen augurken meer.")
  \end{python}

  Een lijst heeft ook een bepaalde vorm van waarheid ingebakken. Zo is een lege
  lijst vals-achtig en een lijst met elementen waar-achtig:
  \begin{python}
    if not (toespijs or groenten):
      print("Je wilt een broodje zonder iets op?!")
  \end{python}

  Om te zien wat je nog allemaal kan doen met lijsten kan je \py{dir([])}
  uitvoeren. Zo vraag je aan Python ``Wat kan ik allemaal doen met een lege
  lijst?''

\section{Strings}
  Eigenlijk ken je nog een andere soort collectie: de string. Een string is een
  geordende verzameling van karakters. Het lijkt heel erg op een lijst. Het
  enige verschil is dat ze onveranderlijk zijn, net zoals de reeks.


\section{Lussen}
  %TODO
  \begin{python}
    # Veronderstel dat we zeker zijn dat len(ingredienten) >= 2
    bestelling = "Voor mij een broodje met "
    for ingr in ingredienten[:-2]:
      bestelling += ingr + ", "
    bestelling += "en " + ingredienten[-1] + " graag."
  \end{python}

  % \subsection{while}:
  % \subsection{break en continue}:

\section{Pointers (extra)}\label{sec:pointers}
