% -*- mode: Latex; ispell-dictionary: "dutch"; -*-
\title{\vspace{-8ex}Programmeren met Python}
\author{}
\date{\vspace{-7ex}2017-2018}
\maketitle

\begin{center}
  \vspace{-3em}
  \begin{table}[htb!]
    \centering
    \begin{tabular}[h]{ll}
      Laurens Bloemen & \anemail{laurensbloemen@telenet.be}\\
      Rugen Heidbuchel & \anemail{rugenheidbuchel@gmail.com} \\
      Menno Vanfrachem & \anemail{mennovanfrachem@hotmail.com}
    \end{tabular}
  \end{table}
  \vspace{-1em}
  Mail ons met al je vragen over programmeren. Vragen over de lessen krijgen
  natuurlijk voorrang!
\end{center}

\section{Programmeeromgeving}
We gebruiken \textbf{\emph{Python 3}}. Opgelet, er is een groot verschil met
\emph{Python 2}!

In de eerste lessen maken we gebruik van de website
\href{https://repl.it/languages/python3}{repl.it} om onze programma's te
schrijven. De website splits op in twee delen. De linkerkant dient om volledige
programma's in te schrijven en uit te voeren, terwijl het rechterdeel enkel lijn
per lijn wordt uitgevoerd en interactief is. Dit rechterdeel gebruik je dus
vooral om je code te testen. Het interactieve deel heet een \emph{REPL} wat
staat voor \emph{read-evaluate-print-loop}. Dat is wat het doet: het leest wat
je typt, voert die code uit, print de uitkomst en begint opnieuw!