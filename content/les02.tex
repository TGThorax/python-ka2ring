% -*- mode: Latex; ispell-dictionary: "dutch"; -*-
\section*{Enkele oefeningen}
  We geven enkele oefeningen op wat we de vorige weken gedaan hebben. De
  oplossingen staan zoals gewoonlijk op
  \href{https://github.com/TGThorax/python-ka2ring/tree/master/src}{github}.
  \begin{itemize}
  \item Maak een programma dat de gebruiker zijn of haar lengte in cm en gewicht
    in kg vraagt en dan als output de bmi van de persoon tot op twee cijfers na
    de komma geeft. De bmi wordt gegeven in eenheden $\nicefrac{kg}{m^2}$.
    Vergeet dus niet de lengte naar meter om te zetten. Ga ook na of de
    opgegeven data wel kan, dus niet kleiner is dan nul.
  \item Vraag de gebruiker om een jaar en ga na of dat een schrikkeljaar is.
    Een schrikkeljaar is een jaar dat deelbaar is door vier, maar is dat niet
    niet wanneer het deelbaar is door honderd, maar weer wel wanneer het
    deelbaar is door vierhonderd.
  \end{itemize}
  Hieronder bekijken we iets nieuws: functies. Functies ken je normaal al uit de
  wiskunde. Bij programmeren hoeven functies helemaal niets wiskundig te
  doen (voor zover een computer niet een grote rekenmachine is natuurlijk). Het
  extra deel over recursie is desalniettemin wiskundig gegeven.
\section{Functies}
  Op dit moment ging ons programma steeds naar beneden: lijnen die laatst
  komen worden laatst uitgevoerd. Soms is het handig om terug naar boven te
  springen naar een bepaald punt, wat berekeningen uit te voeren en dan terug
  te gaan naar waar je vandaan komt. Dat zijn functies. Functies labelen dus
  stukjes code die je later opnieuw kan gebruiken. De syntax voor een functie
  is als volgt:
  \begin{python}
    def naam(argumenten, met, komma, gescheiden):
      # code die je functie uitvoert
      return var # Geef een waarde als uitkomst.
                 # Dat hoef je niet te doen.
      # code na een `return' wordt niet uitgevoerd.
  \end{python}
  Hier is \py{def} een verkorting van deini\"eren in het Engels: define.

  In de vorige voorbeelden moesten we vaak gebruikers naar een cijfer vragen.
  We hadden code in de volgende vaak voorkomende vorm:
  \begin{python}
    nummer = int(input("Vraag aan de gebruiker: "))
  \end{python}
  We kunnen dat een beetje verkortten door een functie te schrijven die die
  \py{int(input(tekst))} automatisch doet voor ons. We noemen die functie
  \py{getal} voor de gemakkelijkheid:
  \begin{python}
    def getal(tekst):
      waarde = int(input(tekst))
      return waarde
  \end{python}
  Hierna kunnen we gewoon de functie \py{getal} gebruiken om onze getallen aan
  de gebruikers op te vragen. Een functie oproepen doe je met
  \py{naam(argument1, argument2, ...)}, net hetzelfde dus als in de definitie,
  alleen nu zonder \py{def} en dubbelepunt op het einde. Om een jaartal te
  vragen is dat dus bijvoorbeeld:
  \begin{python}
    jaar = getal("Geef een jaar: ")
  \end{python}
  Dat is al ietsje korter. Dit was natuurlijk maar een gemakkelijk voorbeeld,
  maar het geeft wel de kracht van functies weer: je moet enkel de naam weten en
  hoe je het moet gebruiken om alle code die erachter zit te kunnen gebruiken.
  Die code zelf hoef je nooit gezien te hebben enkel de naam en instructies. Je
  hebt misschien gemerkt dat de manier waarop we de functie \py{getal} oproepen
  net hetzelfde is als hoe we \py{int}, \py{str}, \py{round}, \py{help},
  \py{dir}, \py{print}, \py{input}, \ldots oproepen. Dat is omdat het elk ook
  stuk voor stuk functies zijn! Je hebt dus al de hele tijd functies gebruikt
  zonder het te weten. Meer nog, je hebt de code achter bijvoorbeeld \py{print}
  nog nooit gezien, maar je hebt ze wel al in bijna elk programma gebruikt. Dat
  is de kracht van functies.

  Noot: de meeste functies die we tot nu gebruikten zijn wel \emph{primitieve}
  functies. Dat zijn functies die in python ingebakken zitten en dus niet in de
  taal zelf ge\"implementeerd zijn.
  Net zoals we hierboven \py{getal} implementeerden, zijn vele
  functies, zoals bijvoorbeeld alle functies in \py{math}, in python opgebouwd
  uit het nauw samenwerken van andere, eerder gedefinieerde functies. Er komt
  natuurlijk een punt wanneer we de eerste functies moeten definieren. Dat zijn
  de primitieve functies. Uit die primitieve functies is de hele taal opgebouwd.

\section{Recursie (extra)}
  Een functie kan je implementeren in termen van zichzelf. Als voorbeeld nemen
  we de faculteit uit de wiskunde, aangeduid met een uitroepteken. Daarna
  berekenen we de Fibonacci getallen. Een klassiek voorbeeld is ook de grootste
  gemene deler.
  \subsection{Faculteit} is als volgt gedefinieerd, met $n \in \mathbb{N}$:
  \begin{equation*}
    n! = \text{fac}\left(n\right) =
    \begin{cases}
      1                                & \text{als}\ n = 0 \\
      n\cdot\text{fac}\left(n-1\right) & \text{anders}
    \end{cases}
  \end{equation*}
  Dus bijvoorbeeld
  $5!=5\cdot(4!)= \ldots = 5\cdot4\cdot3\cdot2\cdot1\cdot1 = 120$. In python
  kunnen we dat als volgt definieren:
  \begin{python}
    def fac(n):
      if (n == 0): return 1
      else: return n * fac(n - 1)
  \end{python}
  Nu kan je gemakkelijk bijvoorbeeld $50!$ uitrekenen zonder dat je alle
  getallen moet uittypen en vermenigvuldigen. Python doet dat voor jou! Als je
  dit met grotere en grotere getallen probeert zal je zien dat de faculteit heel
  erg snel groeit: $1!=1,\ 2!=2, 3!=6, 4!=24, 5!=120, 6!=720, 7!=5040, \ldots, 20!
  \approx 2,4\cdot10^{18}, \ldots, 100! \approx 9,3\cdot10^{157}$. Zoals je ziet
  groeit de macht zelfs sneller dan het getal zelf! In wiskundige woorden:
  voor $n \in \mathbb{N}$ groot genoeg geldt $c^n\ll n!$ voor elke mogelijke
  $c$. Hierbij betekent het symbool $\ll$ `veel kleiner dan'. Natuurlijk moet
  het getal $n$ groot zijn voordat dat waar is. Zo is bijvoorbeeld $10! =
  3628800 < 10^{10}$. De faculteit wordt groter dan de macht op het moment dat
  $n \geq 25$ als $c = 10$. Voor andere getallen $c$ ligt die grens natuurlijk
  ergens anders. Aan jou om die uit te zoeken!
  \\[1em]
  Als je nog grotere getallen neemt, dan zal Python op een gegeven moment
  reclameren. Dat komt omdat python ook steeds moet onthouden naar waar er moet
  teruggekeerd worden nadat de functie afgelopen is. Als je bijvoorbeeld
  \py{fac(10)} berekent, dan moet python 10 keer de functie \py{fac} oproepen.
  Voor 10 valt dat nog mee, maar als je grote getallen neemt verbruikt de
  computer steeds meer geheugen om te onthouden naar waar er moet teruggekeerd
  worden na elke stap. Op dit soort geheugen staat een limiet. Daarom krijg je
  een \emph{exception}, een uitzondering dus, die zegt dat Python te diep voor
  je moet gaan om tot een antwoord te komen. Het is dus niet het getal dat te
  groot wordt voor je computer, maar Python die je programma tijdig stopt.

  Hetzelfde probleem hebben we wanneer we \py{fac} met negatieve getallen
  oproepen. Dan zal python steeds dat getal verlagen totdat het nul bereikt,
  maar het is al lager dan nul. Het zal dus nooit ophouden! Python stopt
  gelukkig tijdig je programma.

  Deze limieten zullen we binnenkort omzeilen door het op een andere manier te
  implementeren.

  \subsection{De Fibonacci getallen} zijn gelijkaardig aan de faculteit, maar
  dan met een \py{+} in plaats van een \py{*} en in paren gedefinieerd:
  \begin{equation*}
    \text{fib}\left(b\right) = %
    \begin{cases}
      0                                                       & \text{als}\ n=1 \\
      1                                                       & \text{als}\ n=2 \\
      \text{fib}\left(n-1\right) + \text{fib}\left(n-2\right) & \text{anders}
    \end{cases}
  \end{equation*}
  Implementeren in Python is gemakkelijk:
  \begin{python}
    def fib(n):
      if n == 1: return 0
      elif n == 2: return 1
      else: return fib(n - 1) + fib(n - 2)
  \end{python}

  Hier zien we wel dat Python per stap vaak \py{fib} moet uitvoeren: voor $n=1$
  en $n=2$ maar \'e\'en keer, voor $n=3$ al drie keer, voor $n=4$ al vijf keer,
  voor $n=5$ negen keer, \ldots. Dat stijgt dus heel erg snel. Python moet dus
  veel rekenwerk verrichten! Het is dan ook niet verwonderlijk dat als je
  bijvoorbeeld het $40$ste Fibonacci getal vraagt aan de computer je toch wel
  even moet wachten voordat Python de uitkomst weet.

  We doen ook heel vaak de zelfde berekeningen. Dat kunnen we zien als we de
  code een beetje aanpassen, zodat het rapporteert wat het op dat moment juist
  bezig is:
  \begin{python}
    def fib(n):
      print("fib(" + str(n) + ")", end=", ") # zie help(print)
      if n == 1:   return 0
      elif n == 2: return 1
      else:        return fib(n - 1) + fib(n - 2)
  \end{python}
  \begin{python}
    fib(5)
    #= fib(5), fib(4), fib(3), fib(2), fib(1),
    #=   fib(2), fib(3), fib(2), fib(1),
    #=> 3
  \end{python}
  Om \py{fib(5)} uit te rekenen moet eerst \py{fib(4)} en dan \py{fib(1)}
  uitgerekend worden. Maar om \py{fib(4)} uit te rekenen moet eerst [\ldots].
  Daarom krijgen we bovenstaande reeks als uitvoeringsvolgorde.
  Zo vaak hetzelfde doen is natuurlijk niet erg effici\"ent. Dat zullen we later
  oplossen.

  \subsection{De grootste gemene deler}
  van twee getallen is het grootste gehele getal waardoor je die twee kan delen
  om een geheel getal te krijgen. Bijvoorbeeld: de grootste gemene deler van
  $25$ en $10$ is $5$, of dus \py{ggd(25, 10) == 5}. Euclides, een wiskundige
  uit de tijd van de Grieken, heeft een algoritme om de grootste gemene deler
  tussen twee getallen te vinden:

  ``De grootste gemene deler van getal $a$ en $b$ is hetzelfde als de grootste
  gemene deler van $b$ en de rest van de deling van $a$ door $b$. Als die rest
  nul is dan is de grootste gemene deler $a$.'' In ons voorbeeld is $a = 25$ en $b
  = 10$, dus is de grootste gemene deler hetzelfde als die van $10$ en $(25\
  \text{rest}\ 10) = 5$. Dat is op zich weer de grootste gemene deler tussen $5$
  en $(10\ \text{test}\ 5) = 0$. Hier stopt het algoritme met de uitkomst $5$. De
  grootste gemene deler tussen $25$ en $10$ is dus $5$. De implementatie is een
  \href{https://github.com/TGThorax/python-ka2ring/blob/master/src/ggd.py}{oefening}.
\subsection{Algemene vorm voor recursieve functies}:
  zoals je in de vorige programma's gezien hebt is het soms mogelijk dat je
  programma oneindig lang zal door gaan. Het is bij recursieve algoritmes dus
  heel belangrijk dat je de juiste stopvoorwaarden gebruikt. Bij faculteit
  hebben we \'e\'en stopvoorwaarde, namelijk als $n=0$. Zoals gezien is dat niet
  genoeg, want de gebruiker kan negatieve getallen opgeven. Een stopvoorwaarde
  $n<0$ toevoegen verhelpt dat probleem. Bij recursieve functies zetten we
  traditioneel de stopvoorwaarde bovenaan in de functie, zodat iemand die de
  code leest direct snapt wanneer de code stopt.
\section{Documentatie}
  Het is altijd een goed idee om je code te documenteren. We weten al dat we
  commentaren kunnen toevoegen met het hekje (\py{\#}). Het is een nog beter
  idee om de functies die je schrijft te documenteren en bij die documentatie
  instructies te verlenen. Als je dat doet kan een gebruiker gewoon
  \py{help(jouw\_functie)} evalueren om jouw instructie te kunnen lezen. Om
  documentatie bij een functie toe te voegen gebruikt je in Python de volgende
  syntax:
  \begin{python}
    def een_functie(arg, um, en, ten):
      "Informatie voor de gebruiker komt hier"
      # Jouw code
    help(een_functie)
    #= help on function een_functie in module een_module
    #= een_functie(arg, um, en, ten)
    #=     Informatie voor de gebruiker komt hier
  \end{python}
  Zoals je ziet kan een goede naamgeving aan de argumenten mogelijk een grote
  hulp zijn aan de gebruiker van je functie.\\
  Noot: Als je een string wilt die over meerdere lijnen heen gaat kan je
  driedubbele aanhalingstekens gebruiken: \py{\char`\"\char`\"\char`\"Tekst met
    meerdere lijnen mogelijk\char`\"\char`\"\char`\"}.
  Dit werkt  niet in een repl, enkel in een programma.
\section{Bereik van variabelen}
  Als je een variabele gebruikt in een functie, dan is die enkel daar bruikbaar.
  De volgende code illustreert:
  \begin{python}
    a = 3
    print("Helemaal vanvoor: a=" + str(a))
    def f():
      a = 2
      print("Erin: a=" + str(a))
    f()
    print("Helemaal erachter: a=" + str(a))

    #= Helemaal ervoor: a=3
    #= Erin: a=2
    #= Helemaal erachter: a=3
  \end{python}
  Zoals je ziet heeft noch de definitie van de functie, noch het oproepen ervan
  iets veranderd aan onze variabele \py{a}. Dat komt omdat eigenlijk de \py{a}
  binnen de functie en die eruit twee verschillende variabelen zijn. De makers
  van python hebben er bewust voor gekozen dat dit zo werkt. Als dit niet het
  geval zou zijn bestaat er een kans dat de variabelen die je mooi klaarzet
  voordat je een functie oproept er na die functie misschien helemaal anders
  uitzien. Stel bijvoorbeeld dat \py{input} in zijn interne code een variabele
  \py{text} aanmaakt. Als je dan code hebt die ook die variabele \py{text} heeft
  en je roept dan de \py{print}functie op, dan kan je het wel stellen met wat je
  in die variabele had opgeslagen. Om zulke dingen te vermijden zorgt Python er
  dus voor dat dat niet kan gebeuren door ervoor te zorgen dat variabelen binnen
  en buiten de functie niet hetzelfde zijn, alhoewel ze qua naam misschien exact
  op elkaar lijken. De regio waar een bepaalde naam of symbool ook hetzelfde
  betekent heet \emph{het bereik van de variabelen} (Engels: scope).
