% -*- mode: Latex; ispell-dictionary: "dutch"; -*-

\section*{Control Flow of besturingsstroom}
  Tot nu toe hebben onze programma's altijd rechtlijnig uitgevoerd. Eerst het
  eerste lijntje code, daarna het volgende, enzovoort. Maar soms willen we dat
  de code een regeltje overslaat of sommige dingen enkele keren doet.
\section{Voorwaardelijk uitvoeren}
  We zullen als voorbeeld het programma dat je leeftijd geeft verbeteren, door
  ook naar de geboortemaand te vragen. Met die maand kunnen we nakijken of de
  persoon nu \py{leeftijd} of \py{leeftijd - 1} jaar oud is. We zullen om te
  beginnen de geboortedag dus achterwege laten. Het programma moet op een moment
  wel de beslissing kunnen maken tussen \py{leeftijd} en \py{leeftijd - 1}. Wij
  zullen de computer moeten zeggen hoe dat juist gebeurt:
  \subsection{if}: in python kan je code voorwaardelijk uitvoeren door \py{if}
  te gebruiken. De syntax, de grammatica dus, hiervoor is als volgt:
  \begin{python}
    if TEST:
      # Alle lijnen met een indentatie (spaties of tabs)
      # worden enkel uitgevoerd als TEST waar is.
    # Vanaf de volgende lijn die terug op de zelfde `hoogte'
    # als de `if' zelf staat, wordt alles terug uitgevoerd.
  \end{python}
  Nu zien we dus ook waarom we geen spaties aan het begin van de lijn mogen
  zetten als we niet bijvoorbeeld \py{if} gebruiken. Wanneer python zo'n
  ge\"indenteerde lijn ziet, zoekt ie automatisch bij welke \emph{blok} dat
  juist hoort. Wanneer iets fout ge\"indenteerd is weet python niet wanneer dat
  stukje code moet uitgevoerd worden, dus bij welke blok het lijntje hoort.

  \subsection{if..else}: Soms moeten we code uitvoeren van de vorm
  \begin{python}
    if TEST:
      # Code als TEST waar is
    if (not TEST):
      # Code als TEST niet waar is
  \end{python}
  Omdat we bij het programmeren vaak ook zo weinig mogelijk in herhaling willen
  vallen is er ook een kortere versie hiervan:
  \begin{python}
    if TEST:
      # Code als TEST waar is
    else:
      # Code als TEST niet waar is
  \end{python}
  Deze code is ook beter, want nu moet Python niet \py{not TEST} uitvoeren
  om te weten of dat tweede stukje moet uitgevoerd worden of niet. Python weet
  immers al dat \py{not TEST} waar zal zijn als \py{TEST} waar is.

  Hieronder verbeteren we de code:
  \begin{python}
    jaar  = int(input("In welk jaar ben je geboren? "))
    maand = int(input("En in welke maand? (getal) "))
    if (maand <= 10): # 10 voor oktober
      # De gebruiker is al verjaard dit jaar
      print("Je bent " + str(2017 - jaar) + " jaar oud.")
    else:
      print("Je bent " + str(2017 - jaar - 1) + " jaar oud.")
  \end{python}
  Dit uitbreiden voor een dag kan al vanaf dit punt, maar zoals we in de lessen
  zagen is dat niet zo gemakkelijk als het lijkt. Het was zo moeilijk omdat we
  het overzicht niet goed konden bewaren over wat we nu juist bezig waren

\section{Mooie code}
  Bovenstaande programma heeft wel een nadeel: het werkt alleen maar juist in
  oktober 2017. Nog erger: wanneer iemand de code wil repareren moet iemand
  eerst zoeken wat er juist fout is. In dit geval zou dat het jaar en de maand
  zijn. Om diegene die de code moet herstellen (dat kan je toekomstige zelf
  zijn) te helpen zullen we die \emph{magische constanten} \py{2017} en \py{10}
  \addcontentsline{toc}{subsection}{Geen magische constanten}%
  een duidelijke naam geven. Om het programma altijd werkende te houden zouden
  we bijvoorbeeld de gebruiker naast zijn of haar verjaardag ook naar de
  huidige dag te vragen. Wat ook kan is de computer de huidige dag zelf laten
  invullen. Dat zou allemaal makkelijker zijn als er maar \'e\'en plaatsje is dat
  aangepast moet worden.

  \addcontentsline{toc}{subsection}{Weinig herhaling}%
  In bovenstaande code zien we ook dat we in de herhaling vallen: het
  enige dat verandert in de twee voorwaarden is de leeftijd. Alle tekst er rond
  blijft hetzelfde. Als we dit programma moeten verbeteren op taalfouten of
  bijvoorbeeld moeten vertalen, dan is dat veel gemakkelijker om dat op een
  plaats te doen dan op meerdere te gelijk. We kunnen beter de leeftijd
  aanpassen in de voorwaarden en op het einde het resultaat aan de gebruiker
  tonen. De aangepaste voorwaarden zijn dan zoiets als
  \begin{python}
    if (maand <= huidige_maand):
      leeftijd = huidig_jaar - jaar
    else:
      leeftijd = huidig_jaar - jaar - 1
    print("Dan ben je " + str(leeftijd) + " jaar oud.")
  \end{python}
  Hier zien we opnieuw dat we in de herhaling vallen! In beide gevallen moeten
  we \py{huidig\_jaar - jaar} doen. Als we dit gewoon altijd doen en dan de
  leeftijd aanpassen wanneer de gebruiker nog niet verjaard is, dan kan het
  eerste gedeelte van de voorwaarde zelfs weggelaten worden. De volledige code
  is dan als volgt uit:
  \begin{python}
    huidig_jaar   = 2017
    huidige_maand = 10
    jaar  = int(input("In welk jaar ben je geboren? "))
    maand = int(input("En in welke maand? (getal) "))
    leeftijd = huidig_jaar - jaar
    if (maand > huidige_maand):
      # De gebruiker is nog niet verjaard
      leeftijd = leeftijd - 1
    print("Dan ben je " + str(leeftijd) + " jaar oud.")
  \end{python}
  Noot: hierboven definieerde we \py{leeftijd} in termen van \py{leeftijd}. In
  de wiskunde kan \py{x = x - 1} natuurlijk nooit, maar onthoud dat een
  enkele \py{=} niet voor gelijkheid, maar voor toewijzing is. In dit geval ziet
  Python aan de rechterkant van de toewijzing een symbool dat hij kent, namelijk
  \py{leeftijd}, en vervangt dat gewoon met de waarde ervan.

  Deze code is al veel gemakkelijker om aan te passen om bijvoorbeeld de
  geboortedag toe te voegen dan daarvoor. (Oefening. Zie
  \href{https://github.com/TGThorax/python-ka2ring/blob/master/src/leeftijd.py}%
  {(een) oplossing en andere vragen op github})

  Het eerste principe, geen magische constanten, kan je bijna altijd doen. In
  het tweede principe, weinig herhaling, kan je zo ver gaan dat je code
  onleesbaar wordt. Het is veel belangrijker dat je code begrijpbaar is dan dat
  ze kort is, maar hoe langer ze is hoe makkelijker er fouten in sluipen en je
  het overzicht verliest. Je moet dus de balans tussen leesbare en begrijpbare
  code en korte code zien te vinden.

\section{Afkortingen}
  Programmeurs moeten veel typen. Daarom doen ze alles om zo veel mogelijk af te
  korten. Hieronder enkele voorbeelden:
  \vspace{-1.5ex}
  \begin{center}
    \begin{tabular}[h]{ll}
      \multicolumn{1}{c}{\textbf{Dit}} & %
        \multicolumn{1}{c}{\textbf{is hetzelfde als}} \\[0.5ex]
      \py{elif TEST:} & \py{else: if TEST:}           \\[1ex]
      \py{a = (b if TEST else c)} & \begin{minipage}[h]{0.3\linewidth}
                                      \py{if TEST: a = b \\
                                          else:~~ a = c} %~ is space in texttt
                                    \end{minipage}    \\[2ex]
      \py{a += b}     & \py{a = a + b}                \\
      \py{a -= b}     & \py{a = a - b}                \\
      \py{a *= b}     & \py{a = a * b}                \\
      \py{a /= b}     & \py{a = a / b}                \\
      \py{a \%= b}    & \py{a = a \% b}
    \end{tabular}
  \end{center}