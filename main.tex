% -*- mode: Latex; ispell-dictionary: "dutch"; -*-
\documentclass[a4paper]{article}
\usepackage{geometry}
 \geometry{
 a4paper,
 left=15mm,
 right=15mm,
 top=20mm,
}
\usepackage[dutch]{babel}
\usepackage{multicol}
\usepackage{calc}
\usepackage{amsmath,amsthm,amsfonts,amssymb}
%\usepackage{listings}
\usepackage{color,graphicx,overpic}
\usepackage{hyperref}
\usepackage{fancyvrb}

\hypersetup{pdfinfo={
  Title={Programmeren met Python},
  Creator={LaTeX},
  Producer={pdfTeX 3.14159265-2.6-1.40.18},
  Author={Laurens Bloemen},
  Subject={Programmeerlessen Python},
  Keywords={lectures, programming, ring, ka2, lessen, programmeren, python}}}

% Redefine section to use less space
\makeatletter
% \@startsection{<name>}{<level>}{<indent>}{<beforeskip>}{<afterskip>}{<style>}*[<altheading>]{<heading>}
% plus and minus: glue stretching
\renewcommand{\section}{\@startsection{section}{1}{0mm}%
                                {-1ex plus -.5ex minus -.2ex}%
                                {0.5ex plus .2ex}%x
                                {\normalfont\large\bfseries}}
\renewcommand{\subsection}{\@startsection{subsection}{2}{0mm}%
                                {-1ex plus -.5ex minus -.2ex}%
                                {-0.5ex plus .2ex}%x
                                {\normalfont\bfseries}}
\makeatother

\newcommand{\anemail}[1]{\href{mailto:#1}{#1}}

% Requires fancyvrb
\newenvironment{python}%
  {\endgraf\vspace{-0.4em}\Verbatim[xleftmargin=1em]}%
  {\endVerbatim\vspace{-0.2em}}
\newcommand{\py}[1]{\texttt{#1}}
\newcommand{\pyt}[1]{\py{#1}}

% % Requires listings
% \lstset{
%   language=python,
%   xleftmargin=1em,
%   showstringspaces=false,
%   columns=fullflexible,
%   mathescape}
% \lstnewenvironment{python}{}{}
% \newcommand{\py}[1]{\lstinline[]@#1@} % BEWARE: NO USING @ in code! % WILL BREAK!!
% \newcommand{\pyt}[1]{\texttt{#1}} % For text markup within code.

\pagestyle{empty} % Turn off header and footer
\setcounter{secnumdepth}{0} % Don't print section numbers
\setlength{\parindent}{0pt} % No indent at start of paragraph

\begin{document}
\pagenumbering{gobble}
% -*- mode: Latex; ispell-dictionary: "dutch"; -*-
\title{\vspace{-8ex}Programmeren met Python}
\author{}
\date{\vspace{-7ex}2017-2018}
\maketitle

\begin{center}
  \vspace{-3em}
  \begin{table}[htb!]
    \centering
    \begin{tabular}[h]{ll}
      Laurens Bloemen & \anemail{laurensbloemen@telenet.be}\\
      Rugen Heidbuchel & \anemail{rugenheidbuchel@gmail.com} \\
      Menno Vanfrachem & \anemail{mennovanfrachem@hotmail.com}
    \end{tabular}
  \end{table}
  \vspace{-1em}
  Mail ons met al je vragen over programmeren. Vragen over de lessen krijgen
  natuurlijk voorrang!\\
  Een ge\"update versie van dit bestand is terug te vinden op \url{https://github.com/TGThorax/python-ka2ring}.
\end{center}

\section*{Programmeeromgeving}
We gebruiken \textbf{\emph{Python 3}}. Opgelet, er is een groot verschil met
\emph{Python 2}!

In de eerste lessen maken we gebruik van de website
\href{https://repl.it/languages/python3}{repl.it} om onze programma's te
schrijven. De website splits op in twee delen. De linkerkant dient om volledige
programma's in te schrijven en uit te voeren, terwijl het rechterdeel enkel lijn
per lijn wordt uitgevoerd en interactief is. Dit rechterdeel gebruik je dus
vooral om je code te testen. Het interactieve deel heet een \emph{REPL} wat
staat voor \emph{read-evaluate-print-loop}. Dat is wat het doet: het leest wat
je typt, voert die code uit, print de uitkomst en begint opnieuw!

\footnotesize
\begin{multicols}{2}
  \setlength{\premulticols}{1pt}
  \setlength{\postmulticols}{1pt}
  \setlength{\multicolsep}{2pt}
  \setlength{\columnsep}{8pt}
  % \setlength{\columnseprule}{0.25pt}

  % -*- mode: Latex; ispell-dictionary: "dutch"; -*-
\fvset{gobble=4} % For the python environment in inside this environment, gobble
                 % up first 4 chars

\section{Algemeen}
  \subsection{Spaties}: vaak mag je spaties zetten waar je maar wilt, zolang het
  niet in het begin van de lijn is en het symbool niet verandert, dus \newline
  `\py{1 + 1}' is hetzelfde als `\py{1+1}', maar `\py{naam}' is
  niet hetzelfde als \newline `\py{na am}'.

  \subsection{Hoofdletters}: Python is hoofdlettergevoelig, dus \py{math.sqrt}
  is niet hetzelfde als \py{Math.Sqrt}.

  \subsection{Commentaren}:\,\py{\# alles achter een hekje wordt niet uitgevoerd.}

  \subsection{Hulp}: je kan Python om hulp vragen. Met \py{dir} kan je zien wat
  de mogelijkheden zijn, en met \py{help} kan je hulp vragen over die
  mogelijkheid. Zie hieronder voor meer.

\section{Wiskunde}\label{seq:wiskunde}
  Resultaten van berekeningen zijn aangeduid met commentaar \py{\#$\Rightarrow$}.
  \begin{tabular}{l@{\hspace{0.5em}}l@{\hspace{0.5em}}l@{\hspace{0.15em}}l}
    \textbf{operatoren}      & plus (\py +)            & \py{+1 + 1 }         & \py{\#$\Rightarrow$ 2}    \\
                             & min (\py -)             & \py{+1 - -1 }        & \py{\#$\Rightarrow$ 2}    \\
                             & maal (\py *)            & \py{10 * 20 }        & \py{\#$\Rightarrow$ 200}  \\
                             & deling (\py /)          & \py{10 / 2 }         & \py{\#$\Rightarrow$ 5.0}  \\
                             & rest of modulo (\py \%) & \py{10 \% 3 }        & \py{\#$\Rightarrow$ 1}    \\
                             & machten (\py{**})       & \py{10 ** 3 }        & \py{\#$\Rightarrow$ 1000} \\
    \textbf{haakjes}         & \py{()}                 & \py{(20 + 3) * 2}    & \py{\#$\Rightarrow$ 46}   \\
    \textbf{kommagetallen}   & punt (\py .)            & \py{3.14 }           & \py{\#$\Rightarrow$ 3.14} \\
    \textbf{gehele getallen} & \py{int}egers           & \py{int(1.2)}        & \py{\#$\Rightarrow$ 1}    \\
                             &                         & \py{int(-1.2)}       & \py{\#$\Rightarrow$ -1}   \\
    \textbf{kommagetallen}   & \py{float}ing point     & \py{float(1)}        & \py{\#$\Rightarrow$ 1.0}  \\
    \\
    \textbf{functies}        & \py{abs}olute waarde    & \py{abs(-10)}        & \py{\#$\Rightarrow$ 10}   \\
                             & vierkantswortel         & \py{math.sqrt(4)}    & \py{\#$\Rightarrow$ 2.0}  \\
                             & machten                 & \py{math.pow(2, 3)}  & \py{\#$\Rightarrow$ 8.0}  \\
    \\
    \textbf{afronden}        &                         & \py{round(4.6)}      & \py{\#$\Rightarrow$ 5}    \\
                             &                         & \py{round(4.56, 2)}  & \py{\#$\Rightarrow$ 4.6}  \\
                             & naar boven   (plafond)  & \py{math.ceil(4.2)}  & \py{\#$\Rightarrow$ 5}    \\
                             & naar beneden (vloer)    & \py{math.floor(4.6)} & \py{\#$\Rightarrow$ 4}
  \end{tabular}
  \\[1em]
  Opgelet: om de functies met \py{math} te kunnen gebruiken moet je Python
  eerst zeggen dat we die zullen gebruiken. Je moet dus eerst \py{import math}
  uitvoeren.

  Er bestaan nog veel meer wiskundige functies die Python kent. Je kan ze
  oplijsten door \py{dir(math)} uit te voeren. Om dan te weten wat bijvoorbeeld
  \py{math.sqrt} juist doet, kan je \py{help(math.sqrt)} uitvoeren.

\section{Toewijzen}
  Python kan ook dingen voor je onthouden. Als je \py{var = 1 + 2} invoert,
  dan slaat Python het getal \py{3} op in de \emph{variabele} \py{var}.
  Die kan je later opnieuw oproepen door gewoon opnieuw dat symbool te
  gebruiken. In dit geval zou \py{var / 3} evalueren naar \py{1.0}.

\section{Logica}
  \begin{tabular}{l@{\hspace{0.5em}}l@{\hspace{0.5em}}l@{\hspace{0.3em}}l}
    \textbf{\emph{Boolean}s} & Waar                        & \pyt{\underline{T}rue}  & \py{\#$\Rightarrow$ True}  \\
                             & Vals                        & \pyt{\underline{F}alse} & \py{\#$\Rightarrow$ False} \\
    \textbf{vergelijken}     & gelijk (\py{==})            & \py{5 == 4.0 + 1}       & \py{\#$\Rightarrow$ True}  \\
                             & niet gelijk (\py{!=})       & \py{5 != 4}             & \py{\#$\Rightarrow$ True}  \\
                             & kleiner (\py{<})            & \py{3 < 2}              & \py{\#$\Rightarrow$ False} \\
                             & kleiner of gelijk (\py{<=}) & \py{2 <= 2}             & \py{\#$\Rightarrow$ True}  \\
                             & groter (\py{>})             & \py{2 > 2}              & \py{\#$\Rightarrow$ False} \\
                             & groter of gelijk (\py{>=})  & \py{2 >= 3}             & \py{\#$\Rightarrow$ False} \\
    \textbf{logica}          & niet (\py{not})             & \py{not True}           & \py{\#$\Rightarrow$ False} \\
                             & en (\py{and})               & \py{(1 < 2) and False}  & \py{\#$\Rightarrow$ False} \\
                             & of (\py{or})                & \py{(1 < 2) or False}   & \py{\#$\Rightarrow$ True}  \\[0.3em]
                             & bevat (\py{in})             & \py{4 in [2, 4, 6]}     & \py{\#$\Rightarrow$ True}  \\
  \end{tabular}
  \\[1em]
  Opgelet: \py{is} bestaat ook in Python, maar is niet hetzelfde als
  \py{==}. Python evalueert \py{1 is 1.0} naar \py{False}!

\section{Tekst en uitvoer}
  Stukjes tekst noemt men \emph{strings}. Je kan ze maken met enkele of dubbele aanhalingstekens.
  \py{"Hallo, ik ben een stukje tekst!"}.
  Tekst kan je ook op het scherm laten verschijnen als output van je programma.
  Dit noemen we \emph{print}en. We zullen uitvoer aanduiden met \py{\#=}.
  Tekst kan je samenvoegen met \py{+} en ook vermenigvuldigen werkt zoals
  verwacht:
  \begin{python}
    tekst = "ik ben een stukje tekst!"
    print("hallo, " * 3 + tekst)
    #= hallo, hallo, hallo, ik ben een stukje tekst!
  \end{python}
  Het bovenstaande stukje code heeft geen resultaat, vandaar dat er geen
  \py{\#$\Rightarrow$} staat. Het heeft wel uitvoer!

  Opgelet: Python weet niet wat er moet gebeuren als je probeert een getal aan
  een string toe te voegen. Als je bijvoorbeeld \py{"test " + 123} uitvoert,
  zegt Python dat hij voor de positie van \py{123} een string verwacht, en
  niet een getal. De juiste manier om getallen bij te voegen is door Python
  eerst te zeggen dat het getal als een stukje tekst moet ge\"interpreteerd
  worden. Dat doe je met \py{str}. De juiste manier om tekst toe te voegen
  is dus: \py{"test " + str(123)}.

\section{Invoer}
  Je kan de gebruiker van je programma ook tekst laten invoeren om die daarna in
  je programma te gebruiken. Dit doe je met \py{input("Instructies voor
    gebruiker hier")}. Natuurlijk moet je programma ook onthouden wat het
  resultaat is, dus moet je het toewijzen. Zo maak je bijvoorbeeld gemakkelijk een
  programmaatje dat je groet:
  \begin{python}
    invoer = input("Hoe heet je?")
    #= Hoe heet je?
    # Het programma wacht tot je iets hebt ingevoerd
    print("Hallo, " + invoer + "!")
    #= Hallo, NAAM!
  \end{python}
  Ik kan de invoer natuurlijk niet voorspellen. Het enige dat ik kan
  schrijven is \py{\#= Hallo, NAAM!}, waar naam de invoer is.

  Opgelet: het resultaat van \py{input}, dus de invoer die je geeft, is
  altijd een string. Als je het als een getal wilt gebruiken, dan moet je eerst
  zeggen tegen Python dat ie het als een getal moet interpreteren. Dat doe je
  met \py{int} of \py{float}. Hieronder een voorbeeld:
  \begin{python}
    antwoord = input("In welk jaar ben je geboren?")
    leeftijd = 2017 - int(antwoord)
    print("Dan ben je nu " + str(leeftijd) + "jaar oud.")
  \end{python}

\section{Een programma schrijven}
  Bovenstaande code bevat een fout! Het kan zijn dat de verjaardag van de
  gebruiker nog niet gevallen is. De juiste leeftijd is dus \py{leeftijd} of
  \py{leeftijd - 1}. Dat kan je programma natuurlijk niet weten, jij hebt
  het immers enkel gezegd dat \py{leeftijd} juist is! Dit illustreert dat
  conceptuele fouten nooit de fout zijn van het programma, maar van de
  programmeur!

\section{Fouten opsporen}
  Wanneer je een fout maakt in de code waardoor Python niet weet wat je bedoelt,
  dan zal Python reclameren. Hieronder enkele van die fouten.

  \subsection{Syntax}: de syntax is de grammatica van de taal. Als de grammatica
  niet juist is, dan weet Python niet wat je bedoelt:
  \begin{python}
    1 + &
    #=     1 + &
    #=         ^
    #= SyntaxError: invalid syntax
    1 + * 1
    #=     1 + * 1
    #=         ^
    #= SyntaxError: invalid syntax
  \end{python}
  Het pijltje duidt aan waar de fout zich juist bevindt.

  \subsection{Naamfouten}: wanneer een symbool of naam niet gedefinieerd is, weet
  Python niet wat je bedoelt:
  \begin{python}
    getal + 1
    #= Traceback (most recent call last):
    #=   File "<stdin>", line 1, in <module>
    #= NameError: name 'a' is not defined
    getal = 10
    getal + 1 #=> 11
  \end{python}
  Hier geeft Python niet zo'n pijltje. Hij zegt wel in welk
  bestand de fout voorkwam en op welke lijn het gebeurde. Hieronder laten we die
  eerste twee lijnen weg, zodat we op de aard van de fout kunnen focussen.

  \subsection{Fout gebruik}: wanneer je een functie fout gebruikt. Gebruik
  \py{help} voor meer info!
  \begin{python}
    abs(1,2)
    #= TypeError: abs() takes exactly one argument (2 given)
    abs(1.2) #=> 1.2
  \end{python}
  Bovenstaande code roept de functie \py{abs} op met twee
  argumenten, namelijk \py{1} en \py{2}, m.a.w. Python ziet de bovenste
  lijn als \mbox{\py{abs(1, 2)}}. Onthoud dat kommagetallen in Python met een punt
  worden geschreven.

  \subsection{Foute types}: De operatie die je wilt doen kan niet met de gegeven
  types:
  \begin{python}
    "20" + 1
    #= TypeError: must be str, not int
    1 + "20"
    #= TypeError: unsupported operand type(s) for +: [..]
    #=            [..] `int' and `str'
    "20" + "1" #=> "201"
    20 + 1     #=> 21
  \end{python}
  Om het type van iets te identificeren kan je altijd \py{type} gebruiken:
  \vspace{-0.5em}
  \begin{center}
    \begin{minipage}{0.46\linewidth}
      \begin{Verbatim}[gobble=8]
        getal = 20
        type(getal)
        #= <class `int'>
        \end{Verbatim}
    \end{minipage} %
    \begin{minipage}{0.46\linewidth}
      \begin{Verbatim}[gobble=8]
        getal = "20"
        type(getal)
        #= <class `str'>
      \end{Verbatim}
    \end{minipage}
  \end{center}

  \subsection{Wiskundige fouten}: bijvoorbeeld delen door nul:
  \begin{python}
    20 / int(0.9)
    #= ZeroDivisionError: division by zero
  \end{python}

  \subsection{Foute code}: soms komt het ook voor dat Python denkt dat ie weet wat
  je bedoelt, maar doet het programma toch iets anders. Een voorbeeld is het
  volgende programma dat een getal vervijfvoudigt:
  \begin{python}
    getal = input("Geef een getal: ")
    print(getal * 5)
  \end{python}
  De variabele \py{getal} is voor Python een string, dus als je dit uitvoert
  en bijvoorbeeld \py{5} als invoer geeft, dan zal het programma
  \py{55555} printen, en niet \py{25}! Je moet dus als programmeur
  opletten met wat voor types variabelen je werkt. In dit geval werken we met
  een \emph{str}ing en niet met een \emph{int}eger.

  \begin{center}{\centering \rule{0.3\linewidth}{0.25pt} Les 00 - 2017-10-17 \rule{0.3\linewidth}{0.25pt}}\end{center}

  % \section{Control Flow of besturingsstroom}
  % Tot nu toe hebben onze programma's altijd rechtlijnig uitgevoerd. Eerst het
  % eerste lijntje code, daarna het volgende, enzovoort. Maar soms willen we dat
  % de code een regeltje overslaat of sommige dingen enkele keren doet.
  % \subsection{if}:
  % \subsection{if .. else}:
  % \subsection{for .. in}:
  % \subsection{while}:
  % \subsection{break en continue}:

  % \section{String formats}
  % \section{Tekstmanipulatie}
  % \section{Je eigen functies}
  % \section{Recursie}
  \clearpage
\end{multicols}
\end{document}
