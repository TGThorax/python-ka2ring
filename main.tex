% -*- mode: Latex; ispell-dictionary: "dutch"; -*-
\documentclass[a4paper,notitlepage]{report}
\usepackage{geometry}
 \geometry{
 a4paper,
 left=15mm,
 right=15mm,
 top=20mm,
}
\usepackage[dutch]{babel}
\usepackage{multicol}
\usepackage{calc}
\usepackage{amsmath,amsthm,amsfonts,amssymb}
\usepackage{nicefrac}
%\usepackage{listings}
\usepackage{color,graphicx,overpic}
\usepackage{hyperref}
\usepackage{fancyvrb}

\hypersetup{pdfinfo={
  Title={Programmeren met Python},
  Creator={LaTeX},
  Producer={pdfTeX 3.14159265-2.6-1.40.18},
  Author={Laurens Bloemen},
  Subject={Programmeerlessen Python},
  Keywords={lectures, programming, ring, ka2, lessen, programmeren, python}}}

% Redefine section to use less space.
\makeatletter
% \@startsection{<name>}{<level>}{<indent>}{<beforeskip>}{<afterskip>}{<style>}*[<altheading>]{<heading>}
% plus and minus: glue stretching
\renewcommand{\section}{%
  \@startsection{section}{1}{0mm}%
  {-1ex plus -.5ex minus -.2ex}%
  {0.5ex plus .2ex}%x
  {\normalfont\large\bfseries}}
\renewcommand{\subsection}{%
  \@startsection{subsection}{2}{0mm}%
  {-1ex plus -.5ex minus -.2ex}%
  {-0.5ex plus .2ex}%
  {\normalfont\bfseries}}
\makeatother
\setcounter{chapter}{-1}
\setcounter{section}{-1}

% Style for the lesson seperator.
\newcommand{\linestyle}[1]{
  \begin{center}
  \mbox{
    \rule{0.25\linewidth}{0.25pt}
    #1
    \rule{0.25\linewidth}{0.25pt}}
  \end{center}}

% args: content, les nr, datum, info
\newcommand{\lesson}[4]{
  \addcontentsline{toc}{chapter}{Les #2 -- #3 -- #4}
  \stepcounter{chapter}
  \setcounter{section}{-1}
  \input{#1}
  \linestyle{Les #2 - #3}
}

\newcommand{\anemail}[1]{\href{mailto:#1}{#1}}

% Requires fancyvrb
\newenvironment{python}%
  {\endgraf\vspace{-0.4em}\Verbatim[xleftmargin=1em]}%
  {\endVerbatim\vspace{-0.2em}}
\newcommand{\py}[1]{\texttt{#1}}
\newcommand{\pyt}[1]{\py{#1}}

% % Requires listings
% \lstset{
%   language=python,
%   xleftmargin=1em,
%   showstringspaces=false,
%   columns=fullflexible,
%   mathescape}
% \lstnewenvironment{pythonl}{}{}
% \newcommand{\pyl}[1]{\lstinline[]@#1@} % BEWARE: NO USING @ in code! % WILL BREAK!!
% \newcommand{\pytl}[1]{\texttt{#1}} % For text markup within code.

\pagestyle{empty} % Turn off header and footer
\setcounter{secnumdepth}{1} % Only sections
\setlength{\parindent}{0pt} % No indent at start of paragraph

\begin{document}
\pagestyle{plain}
% -*- mode: Latex; ispell-dictionary: "dutch"; -*-
\title{\vspace{-8ex}Programmeren met Python}
\author{}
\date{\vspace{-7ex}2017-2018}
\maketitle

\begin{center}
  \vspace{-3em}
  \begin{table}[htb!]
    \centering
    \begin{tabular}[h]{ll}
      Laurens Bloemen & \anemail{laurensbloemen@telenet.be}\\
      Rugen Heidbuchel & \anemail{rugenheidbuchel@gmail.com} \\
      Menno Vanfrachem & \anemail{mennovanfrachem@hotmail.com}
    \end{tabular}
  \end{table}
  \vspace{-1em}
  Mail ons met al je vragen over programmeren. Vragen over de lessen krijgen
  natuurlijk voorrang!\\
  Een ge\"update versie van dit bestand is terug te vinden op \url{https://github.com/TGThorax/python-ka2ring}.
\end{center}

\section*{Programmeeromgeving}
We gebruiken \textbf{\emph{Python 3}}. Opgelet, er is een groot verschil met
\emph{Python 2}!

In de eerste lessen maken we gebruik van de website
\href{https://repl.it/languages/python3}{repl.it} om onze programma's te
schrijven. De website splits op in twee delen. De linkerkant dient om volledige
programma's in te schrijven en uit te voeren, terwijl het rechterdeel enkel lijn
per lijn wordt uitgevoerd en interactief is. Dit rechterdeel gebruik je dus
vooral om je code te testen. Het interactieve deel heet een \emph{REPL} wat
staat voor \emph{read-evaluate-print-loop}. Dat is wat het doet: het leest wat
je typt, voert die code uit, print de uitkomst en begint opnieuw!

\footnotesize
\begin{multicols}{2}
  \setlength{\premulticols}{1pt}
  \setlength{\postmulticols}{1pt}
  \setlength{\multicolsep}{2pt}
  \setlength{\columnsep}{8pt}
  % \setlength{\columnseprule}{0.25pt}

  \lesson{content/les00}{0}{2017-10-17}{Introductie}
  \lesson{content/les01}{1}{2017-10-24}{Control flow: \py{if} en \py{else}}

  % TODO: move to content/les02
  \section*{Enkele oefeningen}
  We geven enkele oefeningen op wat we de vorige weken gedaan hebben. De
  oplossingen staan zoals gewoonlijk op
  \href{https://github.com/TGThorax/python-ka2ring/src}{github}.
  \begin{itemize}
  \item Maak een programma dat de gebruiker zijn of haar lengte in cm en gewicht
    in kg vraagt en dan als output de bmi van de persoon tot op twee cijfers na
    de komma geeft. De bmi wordt gegeven in eenheden $\nicefrac{kg}{m^2}$.
    Vergeet dus niet de lengte naar meter om te zetten. Ga ook na of de
    opgegeven data wel kan, dus niet kleiner is dan nul.
  \item Vraag de gebruiker om een jaar en ga na of dat een schrikkeljaar is.
    Een schrikkeljaar is een jaar dat deelbaar is door vier, maar is dat niet
    niet wanneer het deelbaar is door honderd, maar weer wel wanneer het
    deelbaar is door vierhonderd.
  \end{itemize}
  % \subsection{for .. in}:
  % \subsection{while}:
  % \subsection{break en continue}:

  % \section{String formats}
  % \section{Tekstmanipulatie}
  % \section{Je eigen functies}
  % \section{Recursie}
  \clearpage
  \tableofcontents
  \clearpage
\end{multicols}
\end{document}
